\documentclass[12pt,a4paper]{article}
\usepackage[utf8]{inputenc}
\usepackage[T1]{fontenc}
\usepackage{amsmath}
\usepackage{amssymb}
\usepackage{graphicx}
\usepackage[polish]{babel}
\begin{document}
	\title{Tempomat - symulacja}
	\author{Anna Zalesińska, 155868 \\ Mateusz Juszczak, 155968}
	\date{}
	\maketitle
	\section{Wstęp}
	Celem niniejszego projektu jest implementacja oraz analiza symulatora tempomatu opartego o regulator proporcjonalno-całkująco-różniczkujący (PID). W dzisiejszych czasach, systemy kontroli prędkości, takie jak tempomaty, stanowią integralną część współczesnych pojazdów, wpływając istotnie na komfort jazdy oraz efektywność energetyczną. Regulacja PID, ze względu na swoją prostotę i skuteczność, jest powszechnie stosowaną metodą w projektowaniu systemów sterowania.
	
	W ramach projektu skupiamy się na stworzeniu symulatora tempomatu, który umożliwi zrozumienie oraz ocenę działania regulatora PID w kontekście utrzymania zadanej prędkości pojazdu. Implementacja symulatora pozwoli na eksperymentalne zbadanie wpływu różnych parametrów regulatora PID na stabilność, czas regulacji oraz ogólną wydajność systemu.
	
	Wprowadzenie regulatora PID do układu tempomatu ma na celu eliminację błędów systemowych poprzez odpowiednią korektę sygnału sterującego. Regulator ten składa się z trzech składowych: proporcjonalnej (P), całkującej (I) oraz różniczkującej (D), które łącznie pozwalają na skuteczne utrzymanie pojazdu w zadanym tempie jazdy.
	
	\section{Model matematyczny}
	Parametry determinujące przebieg sumulacji można podzielić na trzy kategorie kategorie: parametry środowiska, parametry obiektu oraz parametry symulacji.
	
	\subsection{Parametry środowiska}
		\begin{center}
			\begin{tabular}{|c|c|c|}
				\hline
				Symbol & Jednostka & Opis \\
				\hline
				\hline
				$func(x)$ & 1 & Funkcja trasy \\
				\hline
				$\rho$ & $\frac{kg}{m^3}$ & Gęstość powietrza \\
				\hline
				$g$ & $\frac{m}{s^2}$ & Przyspieszenie grawitacyjne\\
				\hline
				$v_w$ & $\frac{m}{s}$ & Prędkość wiatru \\
				\hline
				$\alpha_w$ & 1 deg(°) & Kąt wiania wiatru \\
				\hline
			\end{tabular}
		\end{center}
	
	\subsection{Parametry obiektu}
	\begin{center}
		\begin{tabular}{|c|c|c|}
			\hline
			Symbol & Jednostka & Opis \\
			\hline
			\hline
			$A$ & $m^2$ & Powierzchnia czołowa \\
			\hline
			$\mu$ & 1 & Współczynnik oporu toczenia \\
			\hline
			$C_d$ & 1 & Współczynnik oporu aerodynamicznego \\
			\hline
			$m$ & $kg$ & Masa pojazdu \\
			\hline
			$r$ & $m$ & Promień koła \\
			\hline
			$M_l$ & $Nm$ & Maksymalny moment siły silnika \\
			\hline
		\end{tabular}
	\end{center}

	\subsection{Parametry symulacji}
	\begin{center}
		\begin{tabular}{|c|c|c|}
			\hline
			Symbol & Jednostka & Opis \\
			\hline
			\hline
			$e$ & $\frac{m}{s}$ & Uchyb \\
			\hline
			$t$ & $s$ & Czas symulacji \\
			\hline
			$t_s$ & $s$ & Krok symulacji \\
			\hline
			$v_s$ & $\frac{m}{s}$ & Prędkość zadana \\
			\hline
			$v_0$ & $\frac{m}{s}$ & Prędkość początkowa \\
			\hline
			$v$ & $\frac{m}{s}$ & Prędkość chwilowa obiektu \\
			\hline
			$x_0$ & $m$ & Położenie początkowe \\
			\hline
			$\alpha_s$ & 1 deg(°) & Nachylenie zbocza \\
			\hline
		\end{tabular}
	\end{center}
	
	\subsection{Parametry dynamiczne}
	Wartości niektórych parametrów podstawowych są wsprost zależne od aktualnego stanu symulacji. Ich wartości obliczane są podczas każdej iteracji.
	
	\begin{itemize}
		\item Nachylenie zbocza - $\alpha$ \\
		\[
		\alpha = atan \left( \frac{func(x) - func(x - vt_s - 0.1)}{t_s} \right)
		\]
		Parametr jest obliczany jako współczynnik kierunkowy prostej przechodzącej przez punkt aktualnego oraz następnego położenia obiektu z dodatkiem małego, arbitralnie wybranego czynnika skalarnego o wartości 0.1, aby umożliwić poprawne obliczenie tego parametru w sytuacji gdy $v = 0$.
		\item Uchyb - $e$
		\[
			e = v_s - v
		\]
		\item Moment sumulacji - $t_e$
		\[
			t_e = n t_s
		\]
		Gdzie $n$ to liczba wykonanych iteracji symulacji
	\end{itemize}
	
	\subsection{Siły działające na obiekt}
	Podczas symulacji, obiekt poddawany jest wybranym siłom, których wartości są krokowo obliczane. Zależą one zarówno od wartości parametrów ustawionych przez użytkownika, jak i obecnego stanu symulacji.
	\begin{enumerate}
		\item Siła staczania
		\[
			F_s = mgsin\left(\alpha\frac{\pi}{180°}\right)
		\]
		\item Opór toczenia
		\[
			F_t = mg\mu cos\left(\alpha\frac{\pi}{180°}\right)
		\]
		\item Opór wiatru
		\[
			F_w = \frac{1}{2} C_d A \rho v^2
		\]
		\item Opór aerodynamiczny
		\[
			F_a = \frac{1}{2} C_d A \rho (v - v_wcos(\alpha_w\frac{\pi}{180°}))^2
		\]
		\item Siła silnika
		\[
			F_m = max\left(-\frac{M_l}{r} ,min\left(\frac{M_l}{r}, \frac{mkp\left(e(t_e) + \frac{1}{T_i}\int_{-\infty}^{t_s}e(t_i)dt_i + T_d(e(t_e) - e(t_e - t_s))\right)}{t_s}\right)\right)
		\]
		Więcej na temat regulatora w następnej sekcji.
	\end{enumerate}
	Z wymienionych sił obliczana jest siła wypadkowa.
	\[
		F = F_m - F_s - F_t - F_w - F_a
	\]
	
\end{document}