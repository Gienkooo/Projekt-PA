\documentclass[12pt,a4paper]{article}
\usepackage[utf8]{inputenc}
\usepackage[T1]{fontenc}
\usepackage{amsmath}
\usepackage{amssymb}
\usepackage{graphicx}
\usepackage[polish]{babel}
\begin{document}
	\title{Tempomat - symulacja}
	\author{Anna Zalesińska, 155868 \\ Mateusz Juszczak, 155968}
	\date{}
	\maketitle
	
	\begin{center}
		\includegraphics[scale=0.45]{"regulator.png"}
  		$v_0$ - prędkość zadana
    		$v$ - prędkość zmierzona
      		$F_{op}$ - siła oporów
	\end{center}
	\section{Wstęp}
	Celem niniejszego projektu jest implementacja oraz analiza symulatora tempomatu opartego o regulator proporcjonalno-całkująco-różniczkujący (PID). W dzisiejszych czasach, systemy kontroli prędkości, takie jak tempomaty, stanowią integralną część współczesnych pojazdów, wpływając istotnie na komfort jazdy oraz efektywność energetyczną. Regulacja PID, ze względu na swoją prostotę i skuteczność, jest powszechnie stosowaną metodą w projektowaniu systemów sterowania.
	
	W ramach projektu skupiamy się na stworzeniu symulatora tempomatu, który umożliwi zrozumienie oraz ocenę działania regulatora PID w kontekście utrzymania zadanej prędkości pojazdu. Implementacja symulatora pozwoli na eksperymentalne zbadanie wpływu różnych parametrów regulatora PID na stabilność, czas regulacji oraz ogólną wydajność systemu.
	
	Wprowadzenie regulatora PID do układu tempomatu ma na celu eliminację błędów systemowych poprzez odpowiednią korektę sygnału sterującego. Regulator ten składa się z trzech składowych: proporcjonalnej (P), całkującej (I) oraz różniczkującej (D), które łącznie pozwalają na skuteczne utrzymanie pojazdu w zadanym tempie jazdy.
	
	\section{Model matematyczny}
	Parametry determinujące przebieg sumulacji można podzielić na trzy kategorie kategorie: parametry środowiska, parametry obiektu oraz parametry symulacji.
	
	\subsection{Parametry środowiska}
		\begin{center}
			\begin{tabular}{|c|c|c|}
				\hline
				Symbol & Jednostka & Opis \\
				\hline
				\hline
				$func(x)$ & - & Funkcja trasy \\
				\hline
				$\rho$ & $\frac{kg}{m^3}$ & Gęstość powietrza \\
				\hline
				$g$ & $\frac{m}{s^2}$ & Przyspieszenie grawitacyjne\\
				\hline
				$v_w$ & $\frac{m}{s}$ & Prędkość wiatru \\
				\hline
				$\alpha_w$ & 1 deg(°) & Kąt wiania wiatru \\
				\hline
			\end{tabular}
		\end{center}
	
	\subsection{Parametry obiektu}
	\begin{center}
		\begin{tabular}{|c|c|c|}
			\hline
			Symbol & Jednostka & Opis \\
			\hline
			\hline
			$A$ & $m^2$ & Powierzchnia czołowa \\
			\hline
			$\mu$ & - & Współczynnik oporu toczenia \\
			\hline
			$C_d$ & - & Współczynnik oporu aerodynamicznego \\
			\hline
			$m$ & $kg$ & Masa pojazdu \\
			\hline
			$F_{max}$ & $N$ & Maksymalna siła ciągu silnika \\
			\hline
			
		\end{tabular}
	\end{center}

	\subsection{Parametry symulacji}
	\begin{center}
		\begin{tabular}{|c|c|c|}
			\hline
			Symbol & Jednostka & Opis \\
			\hline
			\hline
			$e$ & $\frac{m}{s}$ & Uchyb \\
			\hline
			$t$ & $s$ & Czas symulacji \\
			\hline
			$t_s$ & $s$ & Krok symulacji \\
			\hline
			$v_s$ & $\frac{m}{s}$ & Prędkość zadana \\
			\hline
			$v_0$ & $\frac{m}{s}$ & Prędkość początkowa \\
			\hline
			$v$ & $\frac{m}{s}$ & Prędkość chwilowa obiektu \\
			\hline
			$x_0$ & $m$ & Położenie początkowe \\
			\hline
			$\alpha_s$ & 1 deg(°) & Nachylenie zbocza \\
			\hline
		\end{tabular}
	\end{center}
	
	\subsection{Parametry dynamiczne}
	Wartości niektórych parametrów podstawowych są wprost zależne od aktualnego stanu symulacji. Ich wartości obliczane są podczas każdej iteracji.
	
	\begin{itemize}
		\item Nachylenie zbocza - $\alpha$ \\
		\[
		\alpha = atan \left( \frac{func(x) - func(x - vt_s - 0.1)}{t_s} \right)
		\]
		Parametr jest obliczany jako współczynnik kierunkowy prostej przechodzącej przez punkt aktualnego oraz następnego położenia obiektu z dodatkiem małego, arbitralnie wybranego czynnika skalarnego o wartości 0.1, aby umożliwić poprawne obliczenie tego parametru w sytuacji gdy $v = 0$.
		\item Uchyb - $e$
		\[
			e = v_s - v
		\]
		\item Moment sumulacji - $t_e$
		\[
			t_e = n t_s
		\]
		Gdzie $n$ to liczba wykonanych iteracji symulacji
	\end{itemize}
	
	\subsection{Siły działające na obiekt}
	Podczas symulacji, obiekt poddawany jest wybranym siłom, których wartości są krokowo obliczane. Zależą one zarówno od wartości parametrów ustawionych przez użytkownika, jak i obecnego stanu symulacji.
	\begin{enumerate}
		\item Siła staczania
		\[
			F_s = mgsin\left(\alpha\frac{\pi}{180°}\right)
		\]
		\item Opór toczenia
		\[
			F_t = mg\mu cos\left(\alpha\frac{\pi}{180°}\right)
		\]
		\item Opór wiatru
		\[
			F_w = \frac{1}{2} C_d A \rho v^2
		\]
		\item Opór aerodynamiczny
		\[
			F_a = \frac{1}{2} C_d A \rho (v - v_wcos(\alpha_w\frac{\pi}{180°}))^2
		\]
		\item Siła ciągu silnika
		\[
			F_c = F_{max} \cdot \frac{k_p \cdot \left(e(t_e) + \frac{t_s}{T_i}\sum_{n}^{i=0}e(i \cdot t_s) + T_d \frac{e(t_e) - e(t_e - t_s)}{t_s}\right)}{v_{max}}
		\]
		Więcej na temat regulatora w następnej sekcji.
	\end{enumerate}
	Z wymienionych sił obliczana jest siła wypadkowa.
	\[
		F = F_m - F_s - F_t - F_w - F_a
	\]
	\subsection{Regulator PID}
 	Regulator PID (Proporcjonalny, Całkujący, Różniczkujący) to jeden z najbardziej popularnych rodzajów regulatorów stosowanych w systemach sterowania. Składa się z trzech podstawowych składowych, które są odpowiedzialne za różne aspekty regulacji: \\
  	\begin{itemize}
		\item \textbf{Proporcjonalny (P)} \\ Składowa proporcjonalna reaguje proporcjonalnie do bieżącego błędu między wartością zadaną (celową) a aktualnym stanem systemu. Proporcjonalność oznacza, że im większy jest błąd, tym większa jest korekta wyjścia regulatora. Jednakże, sama składowa proporcjonalna może spowodować, że system będzie zbyt wolno reagował na błędy lub będzie się oscylował wokół wartości zadanego celu. \\
		\item \textbf{Całkujący (I)} \\ Składowa całkująca uwzględnia historię błędów w systemie poprzez sumowanie bieżących błędów w czasie. Dzięki temu, nawet jeśli błąd jest niewielki, ale występuje stale przez dłuższy czas, składowa całkująca stopniowo zwiększa korektę wyjścia regulatora. Składowa ta pomaga eliminować błąd ustalony (czyli różnicę między wartością zadaną a wartością rzeczywistą w stanie ustalonym), jednak może też prowadzić do nadmiernego wzmocnienia i oscylacji. \\
  		\item \textbf{Różniczkujący (D)} \\ Składowa różniczkująca przewiduje przyszłe zmiany poprzez obserwowanie szybkości zmian błędu w czasie. Jej zadaniem jest redukcja oscylacji poprzez hamowanie wzrostu korekty, gdy błąd zmienia się zbyt szybko. Pomaga to w szybszym reagowaniu na zbliżające się wartości zadaną oraz zmniejsza oscylacje wokół tej wartości. Składowa różniczkująca może jednak zwiększać szumy sygnału.\\
	\end{itemize}
 	Regulator PID wykorzystuje kombinację tych trzech składowych, które są mnożone przez odpowiednie współczynniki ($k_p$, $T_i$, $T_d$) i sumowane, aby wygenerować korektę sygnału sterującego. Optymalne dostrojenie (tuning) regulatora PID polega na dobraniu odpowiednich wartości tych współczynników, aby osiągnąć pożądaną stabilność, szybkość reakcji oraz eliminację błędów w systemie regulowanym. Tuning regulatora PID jest często iteracyjnym procesem, wymagającym eksperymentowania z różnymi wartościami współczynników dla danego systemu. \\
 	W przedstawionej implementacji regulatora PID równanie zostało przeskalowane do przedziału wartości $[-1; 1]$, aby reprezentować procent maksymalnej dostępnej siły przyspieszania lub hamowania. Do tego celu wykorzystano stałą $v_{max}$, określającą maksymalną prędkość, jaką pojazd w danej konfiguracji może osiągnąć w warunkach idealnych (bez wpływu wiatru, na poziomej nawierzchni). Wartość $v_{max}$ została obliczona poprzez porównanie siły ciągu z sumą oporów toczenia i aerodynamicznego. Po podstawieniu maksymalnego ciągu oraz innych parametrów pojazdu otrzymuje się wartość maksymalnej prędkości $v_{max}$.
 	\[
 		F_{c} = F_{t} + F_{a}
 	\]	
 	\[
 		F_{c} = mg\mu + \frac{1}{2}C_{d}A\rho v^2
 	\]
 	\[
 		v_{max} = \sqrt{\frac{2F_{max} - 2mg\mu}{C_{d}A\rho}}
 	\]
	\section{Przebieg symulacji}
	Symulator umożliwia manualne dostosowanie parametrów opisanych w sekcji 2 i przedstawia przebieg symulacji na wykresach wybranych wartości symulowanego obiektu. W celu zilustrowania funkcji symulatora, przygotowaliśmy zestaw trzech pojazdów, którym przypisaliśmy odpowiednie parametry obiektu oraz przykładowe ustawienia regulatora.
	
 	\begin{table}[h]
		\centering
		\begin{tabular}{|c|c|c|c|}
			\hline
			Parametry Modelu & VW Passat & Ford Super Duty & VW Up \\
			\hline
			\hline
			Powierzchnia Czołowa [$m^2$] & 2,1 & 2,8 & 1,7 \\
			\hline
			Współczynnik Oporu Toczenia & 0,01 & 0,02 & 0,01 \\
			\hline
			Masa Pojazdu [kg] & 1450 & 2600 & 1080 \\
			\hline
			Maksymalna Siła Ciągu [N] & 833 & 2500 & 403 \\
			\hline
			$K_p$ & 12,5 & 17 & 20,7 \\
			\hline
			$T_i$ & 380 & 215 & 500 \\
			\hline
			$T_d$ & 5 & 2 & 2,18 \\
			\hline
		\end{tabular}
		\caption{Parametry Modeli Pojazdów oraz Nastawy Regulatora }
		\label{tab:transposed_vehicle_parameters}
	\end{table}
	Symulacja pozwala na przeanalizowanie działania tempomatu dla trzech modeli pojazdów oraz czterech kształtów tras. Wspomniane model to VW Passat, Ford Super Duty oraz VW Up. Ich parametry zostały zamieszczone w Tabeli 1.
	Możliwe kształty tras to: teren bez zmian wysokości, arcus tangens, sinus oraz sinus harmoniczny.
	\subsection{Przykład 1 (dobrze nastawiony)}
	Wyniki symulacji dla samochodu VW Passat na trasie arc tan, prędkość zadana 10 m/s o domyślnych nastawach regulatora.
	
	\includegraphics[width=0.9\linewidth]{"zrzut 1.png"}
	\newpage
	\subsection{Przykład 2 (przeregulowany)}
	Wyniki symulacji dla samochodu VW Passat na trasie arc tan, prędkość zadana 10 m/s o nastawach regulatora: \\
	$k_p = 12.5$ \\
	$T_i = 4$ \\
	$T_d = 4.581$ \\
	
	\includegraphics[width=0.9\linewidth]{"zrzut 2.png"}
	\newpage
	\subsection{Przykład 3 (dobrze nastawiony, wymagająca trasa)}
	Wyniki symulacji dla samochodu VW Passat na trasie sinus, prędkość zadana 10 m/s o nastawach regulatora: \\
	$k_p = 12.5$ \\
	$T_i = 380$ \\ 
	$T_d = 5$ \\
	
	\includegraphics[width=0.9\linewidth]{"zrzut 3.png"}
	\newpage
	\subsection{Przykład 4 (dobrze nastawiony)}
	Wyniki symulacji dla samochodu VW Passat na trasie stała, prędkość zadana 10 m/s o domyślnych nastawach regulatora.
	
	\includegraphics[width=0.9\linewidth]{"zrzut 4.png"}
	\newpage
	\subsection{Przykład 5 (porównanie różnych modeli na tej samej trasie)}
	Wyniki symulacji dla predefiniowanych modeli samochodow, każdy regulator domyślnie nastawiony na trasie sinus. \\
	trace 0 - VW Passat \\
	trace 1 - Ford Supet Duty \\
	trace 2 - VW Up \\
	
	\includegraphics[width=0.9\linewidth]{"zrzut 5.png"}
	\newpage
	\subsection{Przykład 6 (wpływ wiejącego wiatru)}
	Wynki symulacji dla VW Passata, domyślne nastawy regulatora, na trasie stałej dla różnych prędkości wiejącego wiatru. \\
	trace 0 - próba kontrolna, brak wiatru \\
	trace 1 - wiatr wiejący w tył pojazdu z prędkością 10 m/s \\
	trace 2 - wiatr wiejący w przód pojazdu z prędkością 10 m/s \\
	
	\includegraphics[width=0.9\linewidth]{"zrzut 6.png"}
\end{document}
